\documentclass[12pt,oneside,a4paper]{book}
\usepackage{graphicx}
\usepackage{polski}
\usepackage[utf8]{inputenc}
\usepackage{hyperref}


\begin{document}
    \begin{titlepage}

        \begin{center}
            \vspace*{1cm}

            \large{
                \textbf{Komunikacja MariaDB - MongoDB}
            }

            \vspace{0.5cm}
            Bazy danych NoSQL
            \linebreak
            niestacjonarne

            \vspace{1.5cm}

            \textbf{Adrian Bury}

            \vfill

            \includegraphics[width=0.4\textwidth]{images/logoup.jpg}

        \end{center}
    \end{titlepage}

    \chapter*{Opracowanie}


    \section*{Algorytm}

    \begin{enumerate}
        \item Uruchomienie serwisów
        \item Zapełnienie bazy MariaDB
        \item Skopiowanie tabeli z bazy MariaDB do bazy MongoDB
    \end{enumerate}

    \noindent
    Podczas wykonywania testu jest zliczany czas potrzebny na
    \begin{itemize}
        \item pobranie elementów z bazy MariaDB
        \item zapisanie elementów do bazy MongoDb
    \end{itemize}

    \section*{Przebieg testu wydajnościowego}

    \begin{itemize}
        \item W bazie MariaDB znajduje się 100 000 rekordów
        \item Baza MongoDB jest pusta (nie są utworzone żadne kolekcje, ani dokumenty)
        \item Test jest uruchamiany 3 razy
    \end{itemize}

    \section*{Wyniki}

    W tabeli zostały przedstawione wyniki wykonanych testów.

    \vspace{0.5cm}

    \noindent
    \begin{tabular}{ l | c | c | c | c }
        Operacja dla 100 000 rekordów & Test nr 1 & Test nr 2 & Test nr 3 & Jednostka \\
        \hline
        Pobranie elementów (MariaDB) & 2.539 & 2.483 & 2.412 & [s] \\
        Dodanie rekordów (MongoDB) & 40.152 & 40.114 & 40.209 & [s] \\
    \end{tabular}

    \section*{Wnioski}
    Baza MariaDB jest w stanie wyciągnąć okokło 40 320 rekordów na sekundę.
    MongoDB jest w stanie zapisywać około
    2 500 rekordów na sekundę.

    \chapter*{Załączniki}
    \begin{itemize}
        \item \href{https://github.com/addun/nosql_2020_niestacjonarne/tree/master/lab1}{Kod źródłowy}
    \end{itemize}

\end{document}