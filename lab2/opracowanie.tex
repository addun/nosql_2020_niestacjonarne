\documentclass[12pt,oneside,a4paper]{book}
\usepackage{graphicx}
\usepackage{polski}
\usepackage[utf8]{inputenc}

\begin{document}
    \begin{titlepage}

        \begin{center}
            \vspace*{1cm}

            \large{
                \textbf{Testy wydajnościowe bazy danych}
            }

            \vspace{0.5cm}
            Bazy danych NoSQL
            \linebreak
            niestacjonarne

            \vspace{1.5cm}

            \textbf{Adrian Bury}

            \vfill

            \includegraphics[width=0.4\textwidth]{opracowanie/images/logoup.jpg}

        \end{center}
    \end{titlepage}

    \chapter*{Opracowanie}


    \section*{Algorytm}

    \begin{enumerate}
        \item Sprawdzenie, czy wszystkie rekordy zostały przeniesione
        \item Losowanie rekordu z bazy nr 1
        \item Sprawdzenie, czy wylosowany rekord istnieje w bazie nr dwa
        \item Jeżeli rekord istnieje w bazie nr 2 powrót do pkt. nr 1
        \item Zapisanie rekordu do bazy nr 2
    \end{enumerate}

    Podczas wykonywania testu jest zliczana liczba operacji, jaka została wykonana na bazach danych
    oraz czas wykonywania jednego przypadku testowego.

    \section*{Przebieg testu wydajnościowego}

    \begin{itemize}
        \item W bazie oznaczonej nr 1 znajduje się 100 rekordów
        \item Baza oznaczona nr 2 jest pusta
        \item Test jest uruchamiany 10 razy
    \end{itemize}

    \section*{Wyniki}

    Na zdjęciu został przedstawiony wykres obrazujący ilość operacji bazodanowych na sekundę w bazie typu
    NoSQL o nazwie MongoDB przy zajętości liczącej sto rekordów.


    \begin{center}
        \makebox[\textwidth]{\includegraphics{opracowanie/images/result.png}}
    \end{center}

    \section*{Podsumowanie}
    Baza danych MongoDB jest w stanie wykonać od 800 do 1 200 operacji na sekundę.
    Ilość ta jest zależna od rodzaju wykonywanych operacji (zapisu, wyszukiwania, czytania).

    \chapter*{Załączniki}
    \begin{itemize}
        \item Kod źródłowy:
    \end{itemize}

\end

\end{document}